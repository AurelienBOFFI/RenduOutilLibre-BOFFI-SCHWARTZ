\documentclass{article}
\usepackage[utf8]{inputenc}
\usepackage[OT1]{fontenc}
\usepackage[french]{babel}
\usepackage{amsmath}
\usepackage{graphicx}
\usepackage[colorlinks=true, allcolors=blue]{hyperref}
\usepackage[section]{placeins}
\usepackage{listings}
\usepackage{array}
\title{Outils libres côté client}
\author{BOFFI Aurélien & SCHWARTZ Nicolas}

\begin{document}
\maketitle
\tableofcontents
\newpage

\section{TP1 - Efficacité de l'environnement}

\subsection{}

\begin{table}[h]
\centering
\begin{tabular}{l|r}

Action + nom logiciel& Raccourci \\\hline
Discord changer de serveur  & CTRL + tab \\\hline
Brave changer d'onglet   & CTRL + tab \\\hline
Brave nouvel onglet   & CTRL + t \\\hline
Brave nouvelle fenetre  & CTRL + n \\\hline
Terminator diviser écran verticalement   & CTRL + SHIFT + e \\\hline
Terminator diviser écran horizontalement   & CTRL + SHIFT + o \\\hline
\end{tabular}
\caption{\label{tab:widgets}Raccourcis.}
\end{table}


\subsection{}

Le site que nous avons trouvé est le site \href{https://10fastfingers.com/typing-test/french}{\emph{Fastfingers}} qui est assez connu.

\begin{figure}[h]
\centering
    \includegraphics[height=0.7\columnwidth]{screen/Fastfingers.png}
    \caption{\label{fig:frog}Exemple score du site}
\end{figure}
\FloatBarrier

\subsection{}
\begin{itemize}
    \item Paramétrer Vim comme éditeur par défaut :
    \begin{itemize}
        \item sudo update-alternatives --config editor
    \end{itemize}
    On choisis ensuite le numéro de Vim et on confirme :
    \begin{figure}
\centering
    \includegraphics[width=\textwidth]{screen/Exo1-3.png}
    \caption{\label{fig:frog}Capture d'écran}
\end{figure}
\end{itemize}
\subsection{}
L'history peut devoiler l'emplacement de fichier sensible, il est donc important de bien le configurer.
Afin d'eviter les trés nombreux ls et cd il faut ajouter :
Dans le ~/.bashrc : HISTIGNORE="ls:cd:history"\\

\subsection{}
Pour la commande mkcd :
Dans le fichier ~/.bashrc ajouter :
\begin{lstlisting}
    function mkcd() { mkdir -p "$@" && cd "$_"; }
    alias mkcd="mkcd"\
\end{lstlisting}
Pour le script git emergency, voici le code :
\begin{lstlisting}
    !#/bin/bash
    gitemergency () {
	git add .
	git commit -m "commit d'urgence" $1
	git push
    }
    \end{lstlisting}
\subsection{}
\begin{itemize}
    \item Créer un script vide backup.sh.
    \item Ajouter le script dans /etc/bash\_completion :
    \begin{lstlisting}
    !#/bin/bash
    _backup() {
    local cur prev opts

    cur="${COMP_WORDS[COMP_CWORD]}"
    prev="${COMP_WORDS[COMP_CWORD-1]}"

    local files=("${cur}"*)

    case $COMP_CWORD in
        1) opts=`getent passwd | cut -d: -f1`;;
        2) opts="now tonight tomorrow";;
        3) opts="${files[@]}";;
        *);;
    esac

    COMPREPLY=()
    COMPREPLY=( $(compgen -W "$opts" -- ${cur}) )
    return 0
    }
    complete -o nospace -F _backup backup
    \end{lstlisting}
\end{itemize}

\subsection{}

\begin{itemize}
\item Pour ajouter le plugin vagrant prompt nous devont nous rendre dans le fichier de notre theme.
\item Les themes sont dans le dossier ~/.oh-my-zsh/themes.
\item Une fois dedans nous devons rajouter les lignes suivantes :

\begin{figure}[h]
    \centering
    \includegraphics[width=0.7\columnwidth]{screen/1_7.png}
    \caption{\label{fig:frog}Capture d'écran.}
    \end{figure}

\end{itemize}

\subsection{}
\begin{itemize}
\item Code à ajouter dans le fichier ~/.zhsrc
\begin{lstlisting}
double-up-history() {
         apache=$(service --status-all | grep apache2| cut -c4)
         echo $apache
         if [[ $apache == "+" ]]
         then
                echo "apache va se stop"
                sudo service apache2 stop
         else
                echo " apache va se lancer"
                sudo service apache2 start
         fi
}

zle -N double-up-history
bindkey "\C-A" double-up-history
\end{lstlisting}
\end{itemize}

\subsection{}

\begin{itemize}
    \item Terminaux testé :Terminator, kitty, coolretroterminal
    \item Terminator : plusieurs onglets ouverts assez facilement + performanant
    \item Kitty : rapide
    \item CoolRetroTerminal : difficile a utiliser en raison de son interface
\end{itemize}
\newpage

\section{TP2 - SSH}
\subsection{}
\begin{itemize}
    \item ssh alice@192.168.56.3
    \item ssh carole@192.168.56.3
    \item ssh bob@192.168.56.3
\end{itemize}
Les history sont vides car c'est la première fois que les machines sont utilisés.

\subsection{}

\begin{itemize}
    \item ssh-keygen
    \item ssh-copy-id alice@192.168.56.3
\end{itemize}
Pour déposer manuellement : 
\begin{itemize}
    \item cat /vagrant/id\_rsa.pub >> ~/.ssh/authorized\_keys 
\end{itemize}

\subsection{}

Ajout de l'host bc :
\begin{itemize}
\item
\begin{lstlisting}
Host bc
   Hostname 192.168.56.3
   User bob
\end{lstlisting}
\end{itemize}


\begin{figure}[h]
\centering
\includegraphics[width=0.7\columnwidth]{screen/ssh3.png}
\caption{\label{fig:frog}SFTP}
\end{figure}

\subsection{}
\begin{itemize}
    \item Commande : 
\begin{lstlisting}
ssh -L 8000:srv.local:80 alice@cli.local 
\end{lstlisting}
\end{itemize}

\subsection{}
\begin{itemize}
    \item Sur notre machine : ssh -D 9000 alice@cli.local
    \item Editer le fichier /etc/tsocks.conf
    \item Commenter les lignes suivantes et modifier la valeur de server\_port :
\begin{lstlisting}
#local = 192.168.0.0/255.255.255.0
#local = 10.0.0.0/255.0.0.0
server_port = 9000

\end{lstlisting}
\end{itemize}

\subsection{}
\begin{itemize}
    \item ssh bc
    \item apt install x11-apps
    \item apt install xauth
    \item \#ForwardX11 yes
    \item ssh -X 192.168.56.3
    \item xeyes
\end{itemize}
\subsection{}
\begin{itemize}
    \item Dans ~/.ssh/config :
\begin{lstlisting}
Host srv
  Hostname 192.168.56.3
  User alice
  ProxyJump cli
#Pour ProxyCommand
 ProxyCommand ssh cli -W%h:%p
\end{lstlisting}
\end{itemize}

\newpage

\section{TP3 - GIT}

\subsection{}


\begin{figure}[h]
\centering
\includegraphics[width=\textwidth]{screen/git1.png}
\caption{\label{fig:frog}Status du début.}
\end{figure}

\begin{figure}[h]
\centering
\includegraphics[width=0.7\columnwidth]{screen/git2.png}
\caption{\label{fig:frog}Le commit.}
\end{figure}

\subsection{}

\begin{figure}[h]
\centering
\includegraphics[width=0.7\columnwidth]{screen/git2_1.png}
\caption{\label{fig:frog}Ajout Patrick et PHP}
\end{figure}

\begin{figure}[h]
\centering
\includegraphics[width=1\columnwidth]{screen/git2_2.png}
\caption{\label{fig:frog}Le commit.}
\end{figure}

\begin{figure}[h]
\centering
\includegraphics[width=0.7\columnwidth]{screen/git2_3.png}
\caption{\label{fig:frog}Tout va bien sur master.}
\end{figure}

\begin{figure}[h]
\centering
\includegraphics[width=0.7\columnwidth]{screen/git2_4.png}
\caption{\label{fig:frog}Commit bien présent.}
\end{figure}

\begin{figure}[h]
\centering
\includegraphics[width=0.7\columnwidth]{screen/git2_5.png}
\caption{\label{fig:frog}Branche supprimé.}
\end{figure}

\subsection{}

\begin{figure}[h]
\centering
\includegraphics[width=0.7\columnwidth]{screen/git3.png}
\caption{\label{fig:frog}Commit bien présent.}
\end{figure}

\begin{figure}[h]
\centering
\includegraphics[width=0.7\columnwidth]{screen/git3_1.png}
\caption{\label{fig:frog}Commit ok.}
\end{figure}

\begin{figure}[h]
\centering
\includegraphics[width=0.7\columnwidth]{screen/git3_2.png}
\caption{\label{fig:frog}Le conflit.}
\end{figure}

\begin{figure}[h]
\centering
\includegraphics[width=0.7\columnwidth]{screen/git3_3.png}
\caption{\label{fig:frog}le conflit.}
\end{figure}

\begin{figure}[h]
\centering
\includegraphics[width=0.7\columnwidth]{screen/git3_4.png}
\caption{\label{fig:frog}Résolution.}
\end{figure}




\end{document}